\chapter{CANopen communication in ROS}
\label{chap:roscanopen}

In this chapter, we will show how to use the ipa\_canopen\_ros library for communicating with CANopen devices.

\section{The tutorial examples}

We have prepared a number of examples that cover the configuration steps which are necessary in order to use the library. These examples can be run via the launch file in their respective directory.
As explained in Section \ref{chap:installation:ipa_canopen_ros}, the examples can be installed as follows:

\begin{itemize}
\item \texttt{git clone \url{git://github.com/ipa320/ipa_canopen_tutorials.git}}
\item \texttt{rosmake ipa\_canopen\_tutorials}
\end{itemize}

At the moment, there are $2 \times 2$ examples in the tutorials. The variations in these examples are:
\begin{itemize}
\item 1 CAN device vs. 3 CAN devices
\item Direct sending of velocity commands to the ipa\_canopen\_ros node vs. using a trajectory controller to drive the device(s) to a desired position.
\end{itemize}

These examples should help you configuring the node for your own setup.

\section{Getting started with ipa\_canopen\_ros}

Before looking into the details, let's first try if everything works.

\begin{itemize}
\item \texttt{roscd ipa\_canopen\_tutorials/examples/1dof\_simple/config}
\item Edit the file \texttt{1dof\_module.yaml} so that it contains a valid CAN id for your hardware (in decimal notation), and the correct device file (e.g. /dev/pcan32).
\item In the same directory, edit the file \texttt{CANopen.yaml}, so that it contains the correct device file and the correct baudrate (e.g. 500K) to which the bus has been configured.
\item \texttt{roslaunch 1dof\_simple.launch}
\item That's all! Now, after the ROS nodes have been launched, you should see your device moving slowly back and forth. If that's the case, your setup has been successful.
\end{itemize}

\section{The canopen\_ros node}

In this section we will describe the \texttt {canopen\_ros} node and the required ROS parameter settings, services, subscribers, and publishers. The node can be run by \texttt{rosrun ipa\_canopen\_ros canopen\_ros}, but will usually be launched from a launch file.

To see tested configuration examples, please refer to the \texttt{config} and \texttt{urdf} subdirectories of the examples in the \texttt{ipa\_canopen\_tutorials} package, and to their corresponding launch files.

\subsection{Parameters}

In this section, we describe the necessary parameter settings. 
\begin{itemize}
\item In the namespace in which the \texttt{canopen\_ros} node is launched, the following parameters need to exist on the ROS parameter server:
  \begin{itemize}
  \item {\bf \texttt{devices}}: A list of CAN device files (\texttt{name}), its corresponding baud rates (\texttt{baudrate}), and SYNC intervals in msec (\texttt{sync\_interval}). Example:
\begin{verbatim}
rosparam get /canopen/devices 
- {baudrate: 500K, name: /dev/pcan32, sync_interval: 10}
- {baudrate: 1M, name: /dev/pcan33, sync_interval: 5}
\end{verbatim}
\item {\bf \texttt{chains}}: A list of chain names (chain = group of devices). Each chain name must correspond to a namespace on the parameter server. Example:
\begin{verbatim}
rosparam get /chains
[arm1_controller, arm2_controller, tray_controller]
\end{verbatim}
  \end{itemize}
\item For each of the chain names listed in parameter \texttt{chains} (cf. above), a namespace of that name must exist with the following parameters:
  \begin{itemize}
  \item {\bf \texttt{joint\_names}}: A list of names/aliases for each joint in your chain.
  \item {\bf \texttt{module\_ids}}: A list of corresponding CAN IDs (in decimal notation).
  \item {\bf \texttt{devices}}: A list of corresponding CAN device files (e.g. /dev/pcan32). The modules in a chain can be on different CAN buses.
  \end{itemize}

\end{itemize}

\subsection{Joint limits (urdf)}

{\bf Note: Work in progress; joint limits and calibration are not yet processed and respected!}

The ros\_canopen node requires at least a rudimentary robot (urdf) model, from which it reads some information regarding the joints (cf. \url{http://www.ros.org/wiki/urdf/XML/joint}) from the \texttt{<joint>} tags. The parameters currently considered are:
\begin{itemize}
\item {\bf \texttt{<limit>}} element {\bf (mandatory)}: \texttt{velocity}, \texttt{lower}, \texttt{upper}
\item {\bf \texttt{<calibration>}} element (optional)
\end{itemize}

You can find example {\em xacro} files in the \texttt{urdf} subdirectories in the examples of \texttt{ipa\_canopen\_tutorials}.

\subsection{Services, Subscribers, and Publishers}

For each chain namespace listed in the \texttt{chains} parameter, the \texttt{ros\_canopen} node subscribes to a required topic, publishes on a number of topics, and exposes the following services:

\subsubsection{Services}

\begin{itemize}
\item {\bf init}: Service datatype \texttt{cob\_srvs/Trigger} (\url{http://ros.org/doc/api/cob_srvs/html/srv/Trigger.html}). This opens the CAN bus connection(s) for the devices in the chain, if not already open, and puts the devices in operational mode.
\item {\bf recover}. Service datatype \texttt{cob\_srvs/Trigger} (\url{http://ros.org/doc/api/cob_srvs/html/srv/Trigger.html}). If devices are in an emergency state, this will recover them to operational mode.
\item {\bf set\_operation\_mode}. Service datatype \texttt{cob\_srvs/SetOperationMode} (\url{http://ros.org/doc/api/cob_srvs/html/srv/SetOperationMode.html}): not used at the moment.
\end{itemize}

\subsubsection{Subscribers}

\begin{itemize}
\item {\bf command\_vel}: Message datatype \texttt{brics\_actuator/JointVelocities} (\url{http://www.ros.org/doc/api/brics_actuator/html/msg/JointVelocities.html}). Essentially a vector of desired velocities for each module in a chain.
\end{itemize}

\subsubsection{Publishers}

\begin{itemize}
\item {\bf state}: Message datatype: \texttt{pr2\_controllers\_msgs/JointTrajectoryControllerState} (\url{http://www.ros.org/doc/api/pr2_controllers_msgs/html/msg/JointTrajectoryControllerState.html}). Vectors of desired and actual positions and velocities of all devices in a chain.
\item {\bf /joint\_states}: Message datatype \texttt{sensor\_msgs/JointState} (\url{http://www.ros.org/doc/api/sensor_msgs/html/msg/JointState.html}). Publisher in the global namespace, taken up e.g. by the \texttt{robot\_state\_publisher} (\url{http://www.ros.org/wiki/robot_state_publisher}).
\end{itemize}

\section{Driving the Schunk LWA 4.6 (Powerball) arm}

The Schunk LWA 4.6 (Powerball) robotic arm can be driven by CANopen commands. You can get the necessary configuration files by cloning the following two repositories:
\begin{itemize}
\item \texttt{git clone git@github.com:ipa320/schunk\_robots.git}
\item \texttt{git clone git@github.com:ipa320/schunk\_modular\_robotics.git}
\end{itemize}

To launch the CANopen driver, together with a trajectory controller:\\
\texttt{roslaunch schunk\_bringup powerball\_solo.launch}

To launch the Powerball arm in Gazebo simulation:\\
\texttt{roslaunch schunk\_bringup\_sim powerball.launch}

This will allow you to control the arm by via JointTrajectoryFollowActions. To test this, you can move to a small number of example configurations from a graphical command GUI:\\
\texttt{roslaunch schunk\_bringup dashboard\_powerball.launch}

\subsection{Inverse kinematics}

{\bf Work in progress}
